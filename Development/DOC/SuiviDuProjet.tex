\documentclass[a4paper,12pt]{article}
\usepackage[francais]{babel}
%\usepackage[latin1]{inputenc}
\usepackage[T1]{fontenc}
\usepackage{textcomp}
\usepackage[utf8]{inputenc}  
\usepackage[T1]{fontenc}  
\usepackage{picins}
\usepackage{wrapfig}
\usepackage{graphicx}
\usepackage[section]{placeins}
\usepackage{lscape}
\usepackage{float} 
\usepackage{amssymb}
\usepackage[nottoc, notlof, notlot]{tocbibind}
\usepackage{eso-pic}
\author{\begin{flushleft}Eté Rémi\end{flushleft}}
\date\today


\voffset -0.95in
\textheight 22cm

%Command just for a background picture
\newcommand\BackgroundPic{
\put(0,0){
\parbox[b][\paperheight]{\paperwidth}{
\vfill
\centering
\includegraphics[width=\paperwidth,height=\paperheight,
keepaspectratio]{   Picture!!!!!   }
\vfill
}}}


\begin{document}

%Add the picture in background
%\begin{center}
%\AddToShipoutPicture{\BackgroundPic}
%\end{center}

\fbox{\begin{minipage}{\textwidth}
\begin{center}
{\LARGE SummerDev Project : \\
Idées et trucs en pagaille}
\end{center}
\end{minipage}}

~\\
~\\

\noindent \underline{Choses indispensables :}
~\\
\begin{enumerate}
\item Gestion des fichiers du projet

\begin{enumerate}
\item Arborescence graphique en widget (sur la gauche de préférence)
\item Toutes les informations du projet dans un fichier de configuration (language, emplacement dans le systeme linux, etc...)
\end{enumerate}

\item Emulation de terminal

\begin{enumerate}
\item Réponse du programme dans le terminal (si programme non graphique)
\item Commandes bash disponibles 
\item Une nouvelle ouverture de terminal ouvre un onglet et pas une nouvelle fenetre indépendante de l'IDE (a réflechir plus en profondeur...)
\end{enumerate}

\item Gestionnaire de make file (C/C++)

\begin{enumerate}
\item Construction automatique des make files a partir du code l'utilisateur (refléchir au linking pas nécéssaire des bibliotheques préfournise)
\item Génération d'un make file unique dans le dossier du projet 
\end{enumerate}

\item Documentation de l'IDE bien construite pour les autres utilisateurs
\item Rendre le projet portable sur une autre machine

\begin{enumerate}
\item Vérifier la compatibilité des variables d'environnement dans les autres architectures Linux
\item Adapter le code au machines Windows
\end{enumerate}

\item Coloration syntaxique du code

\begin{enumerate}
\item La programmation évènementielle fera l'affaire. 
\item Le bout de code devra etre compris de le plug in du language
\end{enumerate}

\item Mathplot en onglet pour du test rapide de graphe et d'outils mathématiques 
\item Coder la gestion des languages sous forme de plug-in

\begin{enumerate}
\item Créer une interface pour chaque language
\item Plug in indépendant du reste de l'IDE (principe du plug in, oui...)
\item Une partie du code sur la GUI
\item Une partie sur la doc
\item Une partie sur la coloration syntaxique (un fichier par coloration)
\item Une partie touchant au configuration du projet
\end{enumerate}

\item Référenciation du code (Réfléchir a la facon de coder ce machin (aucune idée pour l'instant...))

\end{enumerate}

\end{document}
