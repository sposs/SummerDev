\documentclass[a4paper,12pt]{article}
\usepackage[francais]{babel}
%\usepackage[latin1]{inputenc}
\usepackage[T1]{fontenc}
\usepackage{textcomp}
\usepackage[utf8]{inputenc}  
\usepackage[T1]{fontenc}  
\usepackage{picins}
\usepackage{wrapfig}
\usepackage{graphicx}
\usepackage[section]{placeins}
\usepackage{lscape}
\usepackage{float} 
\usepackage{amssymb}
\usepackage[nottoc, notlof, notlot]{tocbibind}
\usepackage{eso-pic}
\author{\begin{flushleft}Eté Rémi\end{flushleft}}
\date\today


\voffset -0.95in
\textheight 22cm


\begin{document}

\fbox{\begin{minipage}{\textwidth}
\begin{center}
{\LARGE SummerDev Project : \\
Modules}
\end{center}
\end{minipage}}

~\\
~\\

%%%% Ressource Module %%%%

\begin{center}
{\LARGE Le module Ressource}
\end{center}

Le module Ressource contient tout les modules indispensables au bon fonctionnement de l'application.
Entre autres :
\begin{itemize}
\item Gestion des threads
\item Suivi de l'installation (et éventuellement de la désinstallation...)
\item Gestion des messages (erreurs, info, checks, etc...)
\item Gestion de base de données (Coloration syntaxique, référenciation du code)
\item Gestionnaire AutoMake
\item ?
\end{itemize}

%%%% Doc Module %%%%

\begin{center}
{\LARGE Le module DOC}
\end{center}

Le module DOC contient tous les modules ou fichier relatif a la documentation. Elle concerne le projet en lui meme ainsi que la documentation de l'IDE. On y trouve donc :
\begin{itemize}
\item La documentation du code source Python de l'IDE
\item La documentation de l'IDE relative à son fonctionnement
\item Une documentation plus générale des modules
\item Le suivi des idées du projet en pdf
\item ?
\end{itemize}

%%%% Env Module %%%%

\begin{center}
{\LARGE Le module ENV}
\end{center}

Le module Env contient toutes informations relatives a l'environnement sur lequel SummerDev est installé.
Entre autre :
\begin{itemize}
\item Les variables d'environnement du système pour l'installation et l'execution de l'IDE
\item ?
\end{itemize}

%%%% Plug-In Module %%%%

\begin{center}
{\LARGE Le module Plug-In}
\end{center}

Le module Plug-In contient les élements indépendants de la structure principale et du fonctionnement principale de l'application.
Pour l'instant :
\begin{itemize}
\item Plug-In SummerPy : Tous les composants pour gérer un projet Python.
\item Plug-In SummerC : Tous les composants pour gérer un projet C.
\item Plug-In SummerCPP : Tous les composants pour gérer un projet C++.
\item Plug-In SummerBash : Tous les composants pour générer un script Bash.
\end{itemize}

En projet :
\begin{itemize}
\item Plu-In SummerMaths : Composants pour gérer le package MathPlot de Python
\item Une serie de tutoriel pour chaque langage (chacun sous forme d'un Plug-In de sorte à gérer les interfaces avec les plug in langage de base)
\end{itemize}

Chaque Plug-In comprendra au moins :
\begin{itemize}
\item Eléments d'interface graphique (GUI)
\item Eléments d'environnement (variables d'environnement pour chaque langage)
\item Des scripts en vrac!!!
\item ?
\end{itemize}

%%%% GUI Module %%%%

\begin{center}
{\LARGE Le module GUI}
\end{center}

Le module GUI contient les composants principaux de la fenetre principale et de tout ce qui n'interagit pas (ou tres peu) au niveau graphique avec le reste de l'application.
Entre autre :
\begin{itemize}
\item La fenetre d'ouverture
\item La fenetre principale
\item Menu principal d'ouverture (Du genre : Ouvrir un projet existant, cliquez sur le projet -> ; Vous ne pinnez rien a la prog? cliquez ici -> )
\item Les fenetres d'information, d'erreurs, de questions, etc...
\item Les fenetres secondaires :
\begin{enumerate}
\item Gestion avancée de compilation avec gcc, g++ et make
\item Chargement de Plug-In (interface avec le module Plug-In)
\item ?
\end{enumerate}
\item ?
\end{itemize}





\end{document}
